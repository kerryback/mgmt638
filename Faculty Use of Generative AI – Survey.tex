% Options for packages loaded elsewhere
\PassOptionsToPackage{unicode}{hyperref}
\PassOptionsToPackage{hyphens}{url}
\PassOptionsToPackage{dvipsnames,svgnames,x11names}{xcolor}
%
\documentclass[
  letterpaper,
  DIV=11,
  numbers=noendperiod]{scrartcl}

\usepackage{amsmath,amssymb}
\usepackage{iftex}
\ifPDFTeX
  \usepackage[T1]{fontenc}
  \usepackage[utf8]{inputenc}
  \usepackage{textcomp} % provide euro and other symbols
\else % if luatex or xetex
  \usepackage{unicode-math}
  \defaultfontfeatures{Scale=MatchLowercase}
  \defaultfontfeatures[\rmfamily]{Ligatures=TeX,Scale=1}
\fi
\usepackage{lmodern}
\ifPDFTeX\else  
    % xetex/luatex font selection
\fi
% Use upquote if available, for straight quotes in verbatim environments
\IfFileExists{upquote.sty}{\usepackage{upquote}}{}
\IfFileExists{microtype.sty}{% use microtype if available
  \usepackage[]{microtype}
  \UseMicrotypeSet[protrusion]{basicmath} % disable protrusion for tt fonts
}{}
\makeatletter
\@ifundefined{KOMAClassName}{% if non-KOMA class
  \IfFileExists{parskip.sty}{%
    \usepackage{parskip}
  }{% else
    \setlength{\parindent}{0pt}
    \setlength{\parskip}{6pt plus 2pt minus 1pt}}
}{% if KOMA class
  \KOMAoptions{parskip=half}}
\makeatother
\usepackage{xcolor}
\setlength{\emergencystretch}{3em} % prevent overfull lines
\setcounter{secnumdepth}{-\maxdimen} % remove section numbering
% Make \paragraph and \subparagraph free-standing
\makeatletter
\ifx\paragraph\undefined\else
  \let\oldparagraph\paragraph
  \renewcommand{\paragraph}{
    \@ifstar
      \xxxParagraphStar
      \xxxParagraphNoStar
  }
  \newcommand{\xxxParagraphStar}[1]{\oldparagraph*{#1}\mbox{}}
  \newcommand{\xxxParagraphNoStar}[1]{\oldparagraph{#1}\mbox{}}
\fi
\ifx\subparagraph\undefined\else
  \let\oldsubparagraph\subparagraph
  \renewcommand{\subparagraph}{
    \@ifstar
      \xxxSubParagraphStar
      \xxxSubParagraphNoStar
  }
  \newcommand{\xxxSubParagraphStar}[1]{\oldsubparagraph*{#1}\mbox{}}
  \newcommand{\xxxSubParagraphNoStar}[1]{\oldsubparagraph{#1}\mbox{}}
\fi
\makeatother


\providecommand{\tightlist}{%
  \setlength{\itemsep}{0pt}\setlength{\parskip}{0pt}}\usepackage{longtable,booktabs,array}
\usepackage{calc} % for calculating minipage widths
% Correct order of tables after \paragraph or \subparagraph
\usepackage{etoolbox}
\makeatletter
\patchcmd\longtable{\par}{\if@noskipsec\mbox{}\fi\par}{}{}
\makeatother
% Allow footnotes in longtable head/foot
\IfFileExists{footnotehyper.sty}{\usepackage{footnotehyper}}{\usepackage{footnote}}
\makesavenoteenv{longtable}
\usepackage{graphicx}
\makeatletter
\newsavebox\pandoc@box
\newcommand*\pandocbounded[1]{% scales image to fit in text height/width
  \sbox\pandoc@box{#1}%
  \Gscale@div\@tempa{\textheight}{\dimexpr\ht\pandoc@box+\dp\pandoc@box\relax}%
  \Gscale@div\@tempb{\linewidth}{\wd\pandoc@box}%
  \ifdim\@tempb\p@<\@tempa\p@\let\@tempa\@tempb\fi% select the smaller of both
  \ifdim\@tempa\p@<\p@\scalebox{\@tempa}{\usebox\pandoc@box}%
  \else\usebox{\pandoc@box}%
  \fi%
}
% Set default figure placement to htbp
\def\fps@figure{htbp}
\makeatother

\KOMAoption{captions}{tableheading}
\makeatletter
\@ifpackageloaded{caption}{}{\usepackage{caption}}
\AtBeginDocument{%
\ifdefined\contentsname
  \renewcommand*\contentsname{Table of contents}
\else
  \newcommand\contentsname{Table of contents}
\fi
\ifdefined\listfigurename
  \renewcommand*\listfigurename{List of Figures}
\else
  \newcommand\listfigurename{List of Figures}
\fi
\ifdefined\listtablename
  \renewcommand*\listtablename{List of Tables}
\else
  \newcommand\listtablename{List of Tables}
\fi
\ifdefined\figurename
  \renewcommand*\figurename{Figure}
\else
  \newcommand\figurename{Figure}
\fi
\ifdefined\tablename
  \renewcommand*\tablename{Table}
\else
  \newcommand\tablename{Table}
\fi
}
\@ifpackageloaded{float}{}{\usepackage{float}}
\floatstyle{ruled}
\@ifundefined{c@chapter}{\newfloat{codelisting}{h}{lop}}{\newfloat{codelisting}{h}{lop}[chapter]}
\floatname{codelisting}{Listing}
\newcommand*\listoflistings{\listof{codelisting}{List of Listings}}
\makeatother
\makeatletter
\makeatother
\makeatletter
\@ifpackageloaded{caption}{}{\usepackage{caption}}
\@ifpackageloaded{subcaption}{}{\usepackage{subcaption}}
\makeatother

\usepackage{bookmark}

\IfFileExists{xurl.sty}{\usepackage{xurl}}{} % add URL line breaks if available
\urlstyle{same} % disable monospaced font for URLs
\hypersetup{
  colorlinks=true,
  linkcolor={blue},
  filecolor={Maroon},
  citecolor={Blue},
  urlcolor={Blue},
  pdfcreator={LaTeX via pandoc}}


\author{}
\date{}

\begin{document}


\section{Faculty Use of Generative AI --
Survey}\label{faculty-use-of-generative-ai-survey}

For this survey, ``generative AI'' includes tools such as ChatGPT,
Claude, Gemini, Copilot, Replit AI, Lovable, GitHub Copilot, etc.

Throughout, check the box if something applies (you ``have used'' it). 
Leave it blank if you have not used generative AI in that way in the
past 12 months.

\subsubsection{Name:
\_\_\_\_\_\_\_\_\_\_\_\_\_\_\_\_\_\_\_\_\_\_\_\_\_\_\_\_}\label{name-____________________________}

\subsubsection{Q1. Assignments that ask students to use generative AI
outside of
class}\label{q1.-assignments-that-ask-students-to-use-generative-ai-outside-of-class}

Instructions:\\
For this question, we are only interested in situations where you
explicitly asked or required students to use generative AI outside of
live class sessions (for homework, projects, preparation, or self-study)
in the past 12 months.

Check the box for each type of activity that you have asked students to
do using generative AI outside of class in at least one of your
courses. If you have never asked students to use generative AI outside of class,
leave all items in this question blank.

\begin{itemize}
\tightlist
\item[$\square$]
  Online research or fact-finding (e.g., exploring a topic, gathering or
  summarizing information, generating examples)\\
\item[$\square$]
  Brainstorming ideas (e.g., project topics, business ideas, marketing
  campaigns, strategies, alternatives)\\
\item[$\square$]
  Drafting or revising written work (e.g., reports, memos,
  presentations)\\
\item[$\square$]
  Data analysis or coding (e.g., writing or debugging code, running
  analyses, creating visualizations -- including use as a coding
  assistant)\\
\item[$\square$]
  Using generative AI as a tutor or study coach (e.g., to explain
  concepts, walk through solutions, or create practice questions)\\
\item[$\square$]
  Preparing for exams (e.g., generating practice questions, study plans,
  or summaries of course material)\\
\item[$\square$]
  Practicing communication outside class (e.g., mock interviews,
  pitches, or role-play with AI)\\
\item[$\square$]
  Asking generative AI to critique or improve their own work (e.g.,
  suggest improvements, alternatives, or corrections)\\
\item[$\square$]
  Other ways I have asked students to use generative AI outside class
  (briefly specify): 
\end{itemize}

\subsubsection{Q2. In-class uses (role-playing, data analysis, chatting,
other)}\label{q2.-in-class-uses-role-playing-data-analysis-chatting-other}

Instructions:\\
In the past 12 months, in any course you taught, check the box for each
way you have used generative AI live during class (in-person or
synchronous online).  If you have not used generative AI in a particular way, leave it blank.

\begin{itemize}
\tightlist
\item[$\square$]
  Chatting / Q\&A in class: Asking an AI tool questions in front of the
  class (e.g., explanations, definitions, worked examples)\\
\item[$\square$]
  Brainstorming with the class: Using AI in class to generate ideas,
  examples, or strategies together with students\\
\item[$\square$]
  Role-playing / simulations: Having AI act as a stakeholder (e.g.,
  customer, manager, regulator, negotiation counterpart)\\
\item[$\square$]
  Data analysis in class: Using an AI tool to run data analysis or code
  live in class (e.g., regression, simulation, visualization)\\
\item[$\square$]
  Coding demonstrations: Using an AI coding assistant in class to write
  or debug code in real time\\
\item[$\square$]
  Generating or modifying cases / scenarios on the fly in response to
  class discussion\\
\item[$\square$]
  Using AI to check or verify answers during class (e.g., checking
  numerical answers or alternative solutions)\\
\item[$\square$]
  Using AI to give feedback on sample student answers during class\\
\item[$\square$]
  Other in-class uses of generative AI (briefly specify):
  \_\_\_\_\_\_\_\_\_\_\_\_\_\_
\end{itemize}

\clearpage 

\subsubsection{Q3. AI policies and
misuse}\label{q3.-ai-policies-and-misuse}

Instructions:\\
Check each statement that is true for at least one of your courses in
the most recent twelve months. Leave any statement blank if it is not true for any of your courses.

\begin{itemize}
\tightlist
\item[$\square$]
  I have an explicit written policy on student use of generative AI in
  at least one course (e.g., in the syllabus or LMS).\\
\item[$\square$]
  I have talked in class about how students may or may not use
  generative AI in at least one course.\\
\item[$\square$]
  I have no explicit policy on student use of generative AI in any of my
  courses.\\
\item[$\square$]
  I have experienced what I believe was student misuse of AI in at least
  one of my courses. Please describe, even if misuse was not definitely
  established. \_\_\_\_\_\_\_\_\_\_\_\_\_\_\_\_
\end{itemize}

\subsubsection{Q4. What you allow or require (learning vs
assignments)}\label{q4.-what-you-allow-or-require-learning-vs-assignments}

Instructions:\\
For each statement below, check the box if it is true of at least one
course you taught in the most recent academic year.  If it is not true for any course, leave it blank.

\begin{itemize}
\tightlist
\item[$\square$]
  I allow students to use generative AI for learning (e.g., summarizing
  readings, asking questions) in at least one course.\\
\item[$\square$]
  I allow students to use generative AI to help with graded written
  assignments in at least one course.\\
\item[$\square$]
  I allow students to use generative AI to help with graded coding or
  data analysis assignments in at least one course.\\
\item[$\square$]
  I explicitly prohibit student use of generative AI for graded
  assignments in at least one course.\\
\item[$\square$]
  I require students to disclose any use of generative AI on graded work
  in at least one course.\\
\item[$\square$]
  I have at least one course where using generative AI is a required or
  central part of the coursework.
\end{itemize}

\clearpage 

For the next questions, we distinguish between:

\begin{itemize}
\tightlist
\item
  Chat-based tools (primarily text/chat interfaces)\\
\item
  Coding / code-execution tools (primarily used for coding or running
  code / data analysis)
\end{itemize}

For each tool, you can mark:

\begin{itemize}
\tightlist
\item
  ``Used myself'' = you have personally used or experimented with this
  tool for research, teaching, or other professional work.\\
\item
  ``Used with students'' = you have used, demonstrated, required, or
  explicitly recommended this tool for students in at least one course.
\end{itemize}

If neither applies, leave both blank.

\subsubsection{Q5. Chat-based tools (primarily
text/chat)}\label{q5.-chat-based-tools-primarily-textchat}

\renewcommand{\arraystretch}{1.5}
\begin{tabular}{@{}>{\raggedright\arraybackslash}p{7.5cm} c c@{}}
\textbf{Tool} & \shortstack{\textbf{Used}\\\textbf{myself}} & \shortstack{\textbf{Used with}\\\textbf{students}} \\
\hline
ChatGPT (standard web or mobile app) & $\square$ & $\square$ \\
Custom GPTs I created or configured & $\square$ & $\square$ \\
Microsoft Copilot (e.g., in Edge, Word, PowerPoint) & $\square$ & $\square$ \\
Google Gemini (chat or in Docs/Sheets/Slides) & $\square$ & $\square$ \\
Anthropic Claude (chat interface) & $\square$ & $\square$ \\
Perplexity AI & $\square$ & $\square$ \\
Other chat-based AI assistant (specify): \_\_\_\_\_\_\_\_\_\_\_ & $\square$ & $\square$ \\
\end{tabular}
\renewcommand{\arraystretch}{1}

\subsubsection{Q6. Coding / code-execution tools (IDE /
agents)}\label{q6.-coding-code-execution-tools-ide-agents}

\renewcommand{\arraystretch}{1.5}
\begin{tabular}{@{}>{\raggedright\arraybackslash}p{7.5cm} c c@{}}
\textbf{Tool} & \shortstack{\textbf{Used}\\\textbf{myself}} & \shortstack{\textbf{Used with}\\\textbf{students}} \\
\hline
Claude Code / OpenAI Codex / Gemini CLI & $\square$ & $\square$ \\
Replit / Lovable & $\square$ & $\square$ \\
VS Code, PyCharm, etc. with GitHub Copilot & $\square$ & $\square$ \\
Cursor / Windsurf / Codeium & $\square$ & $\square$ \\
Gemini AI Studio & $\square$ & $\square$ \\
Julius.ai & $\square$ & $\square$ \\
Other AI coding / code-execution tool (specify): \_\_\_\_\_ & $\square$ & $\square$ \\
\end{tabular}
\renewcommand{\arraystretch}{1}

Thank you for completing this survey.




\end{document}
