% Options for packages loaded elsewhere
\PassOptionsToPackage{unicode}{hyperref}
\PassOptionsToPackage{hyphens}{url}
\PassOptionsToPackage{dvipsnames,svgnames,x11names}{xcolor}
%
\documentclass[
  letterpaper,
  DIV=11,
  numbers=noendperiod]{scrartcl}

\usepackage{amsmath,amssymb}
\usepackage{iftex}
\ifPDFTeX
  \usepackage[T1]{fontenc}
  \usepackage[utf8]{inputenc}
  \usepackage{textcomp} % provide euro and other symbols
\else % if luatex or xetex
  \usepackage{unicode-math}
  \defaultfontfeatures{Scale=MatchLowercase}
  \defaultfontfeatures[\rmfamily]{Ligatures=TeX,Scale=1}
\fi
\usepackage{lmodern}
\ifPDFTeX\else  
    % xetex/luatex font selection
\fi
% Use upquote if available, for straight quotes in verbatim environments
\IfFileExists{upquote.sty}{\usepackage{upquote}}{}
\IfFileExists{microtype.sty}{% use microtype if available
  \usepackage[]{microtype}
  \UseMicrotypeSet[protrusion]{basicmath} % disable protrusion for tt fonts
}{}
\makeatletter
\@ifundefined{KOMAClassName}{% if non-KOMA class
  \IfFileExists{parskip.sty}{%
    \usepackage{parskip}
  }{% else
    \setlength{\parindent}{0pt}
    \setlength{\parskip}{6pt plus 2pt minus 1pt}}
}{% if KOMA class
  \KOMAoptions{parskip=half}}
\makeatother
\usepackage{xcolor}
\setlength{\emergencystretch}{3em} % prevent overfull lines
\setcounter{secnumdepth}{-\maxdimen} % remove section numbering
% Make \paragraph and \subparagraph free-standing
\makeatletter
\ifx\paragraph\undefined\else
  \let\oldparagraph\paragraph
  \renewcommand{\paragraph}{
    \@ifstar
      \xxxParagraphStar
      \xxxParagraphNoStar
  }
  \newcommand{\xxxParagraphStar}[1]{\oldparagraph*{#1}\mbox{}}
  \newcommand{\xxxParagraphNoStar}[1]{\oldparagraph{#1}\mbox{}}
\fi
\ifx\subparagraph\undefined\else
  \let\oldsubparagraph\subparagraph
  \renewcommand{\subparagraph}{
    \@ifstar
      \xxxSubParagraphStar
      \xxxSubParagraphNoStar
  }
  \newcommand{\xxxSubParagraphStar}[1]{\oldsubparagraph*{#1}\mbox{}}
  \newcommand{\xxxSubParagraphNoStar}[1]{\oldsubparagraph{#1}\mbox{}}
\fi
\makeatother


\providecommand{\tightlist}{%
  \setlength{\itemsep}{0pt}\setlength{\parskip}{0pt}}\usepackage{longtable,booktabs,array}
\usepackage{calc} % for calculating minipage widths
% Correct order of tables after \paragraph or \subparagraph
\usepackage{etoolbox}
\makeatletter
\patchcmd\longtable{\par}{\if@noskipsec\mbox{}\fi\par}{}{}
\makeatother
% Allow footnotes in longtable head/foot
\IfFileExists{footnotehyper.sty}{\usepackage{footnotehyper}}{\usepackage{footnote}}
\makesavenoteenv{longtable}
\usepackage{graphicx}
\makeatletter
\newsavebox\pandoc@box
\newcommand*\pandocbounded[1]{% scales image to fit in text height/width
  \sbox\pandoc@box{#1}%
  \Gscale@div\@tempa{\textheight}{\dimexpr\ht\pandoc@box+\dp\pandoc@box\relax}%
  \Gscale@div\@tempb{\linewidth}{\wd\pandoc@box}%
  \ifdim\@tempb\p@<\@tempa\p@\let\@tempa\@tempb\fi% select the smaller of both
  \ifdim\@tempa\p@<\p@\scalebox{\@tempa}{\usebox\pandoc@box}%
  \else\usebox{\pandoc@box}%
  \fi%
}
% Set default figure placement to htbp
\def\fps@figure{htbp}
\makeatother

\KOMAoption{captions}{tableheading}
\makeatletter
\@ifpackageloaded{caption}{}{\usepackage{caption}}
\AtBeginDocument{%
\ifdefined\contentsname
  \renewcommand*\contentsname{Table of contents}
\else
  \newcommand\contentsname{Table of contents}
\fi
\ifdefined\listfigurename
  \renewcommand*\listfigurename{List of Figures}
\else
  \newcommand\listfigurename{List of Figures}
\fi
\ifdefined\listtablename
  \renewcommand*\listtablename{List of Tables}
\else
  \newcommand\listtablename{List of Tables}
\fi
\ifdefined\figurename
  \renewcommand*\figurename{Figure}
\else
  \newcommand\figurename{Figure}
\fi
\ifdefined\tablename
  \renewcommand*\tablename{Table}
\else
  \newcommand\tablename{Table}
\fi
}
\@ifpackageloaded{float}{}{\usepackage{float}}
\floatstyle{ruled}
\@ifundefined{c@chapter}{\newfloat{codelisting}{h}{lop}}{\newfloat{codelisting}{h}{lop}[chapter]}
\floatname{codelisting}{Listing}
\newcommand*\listoflistings{\listof{codelisting}{List of Listings}}
\makeatother
\makeatletter
\makeatother
\makeatletter
\@ifpackageloaded{caption}{}{\usepackage{caption}}
\@ifpackageloaded{subcaption}{}{\usepackage{subcaption}}
\makeatother

\usepackage{bookmark}

\IfFileExists{xurl.sty}{\usepackage{xurl}}{} % add URL line breaks if available
\urlstyle{same} % disable monospaced font for URLs
\hypersetup{
  pdftitle={Equity Fall 2025 },
  colorlinks=true,
  linkcolor={blue},
  filecolor={Maroon},
  citecolor={Blue},
  urlcolor={Blue},
  pdfcreator={LaTeX via pandoc}}


\title{EquityFall
2025\includegraphics[width=0.4\linewidth,height=\textheight,keepaspectratio]{images/RiceBusiness-transparent-logo-sm.png}}
\author{}
\date{}

\begin{document}
\maketitle


\paragraph{Instructor}\label{instructor}

\href{https://kerryback.com}{Kerry Back}
\href{mailto:\%20kerryback@gmail.com}{kerryback@gmail.com} J. Howard
Creekmore Professor of Finance and Professor of Economics

\paragraph{Meeting Schedule}\label{meeting-schedule}

McNair 317 MW 2:15-3:45 10/27/2025 -- 12/08/2025

\paragraph{Course Description}\label{course-description}

The topic of this course is selecting stocks based on quantitative
signals. What we want to do is to predict which stocks will do better
than others. In quantitative investing, this is often done using machine
learning, which means fitting a model to predict returns from signals.
Signals can be technical or fundamental. Even subjective things like the
tone of the CEO in an earnings call are quantifiable today. Thus,
quantitative investing includes a large part of equity investing.

Many different types of data are used for constructing trading signals.
In this course, we will primarily use past returns and corporate
financial statements. Textual analysis of corporate filings and news
reports, social media activity, web traffic, insider trades, short
sales, analyst forecasts, satellite imagery, and other types of data are
also often used in practice.

Many studies have been published on the effectiveness of different
quantitative signals for predicting stock returns. We will discuss those
studies throughout the course and use them to guide our construction of
signals.

\paragraph{Data and Tools}\label{data-and-tools}

We will use data from the
\href{https:/data-portal.rice-business.org}{Rice Business Data Portal},
and we will work with it using Anthropic's Claude models (especially
Sonnet 4.5). We will need Claude Desktop (the web version does not have
one of the features that we will need).
\href{https://claude.ai/download}{Download Claude Desktop} and install
it. When it opens, you will be asked to sign in or to sign up for a new
account. We need the Pro account (\$20/month). You will be reimbursed
for the expense.

To connect the data portal to Claude, download this
\href{rice-stock-data.mcpb}{mcpb file}. Double-click it. If necessary,
right-click and choose \texttt{Open\ with} and then browse to find
\texttt{claude.exe} (on Windows, it is probably in
C:\users\your-name\AppData\Local\AnthropicClaude). Follow the
installation instructions. If that does not work, ask Claude to do a web
search to find out how to install mcpb files and then follow the
instructions.

\paragraph{Assignments and Grading}\label{assignments-and-grading}

Grades will be based on three group assignments, peer assessments, and
class participation.

\begin{enumerate}
\def\labelenumi{\arabic{enumi}.}
\tightlist
\item
  Submit a Word doc containing tests of two portfolio strategies. Both
  strategies should be long-short strategies based on sorting stocks
  into portfolios once per month on the basis of some criterion. One
  strategy should use a criterion derived from fundamentals and the
  other should use a criterion derived from past returns. The tests
  should include (i) a plot of cumulative returns of the portfolios,
  (ii) the Sharpe ratio of the portfolio that goes long one extreme and
  short the other, and (iii) the CAPM alpha of the long-short portfolio.
  It is important to describe the strategies sufficiently clearly that
  your results can be replicated.
\item
  Submit a Word doc containing the result of a backtest of a
  machine-learning-based strategy. The machine learning model should
  combine signals based on fundamentals and past returns to predict
  relative stock returns over the following month. Relative return means
  return minus the median return. The portfolio strategy should sort
  stocks based on predicted relative returns. The model parameters each
  month must be fitted and validated using only data from prior months
  so that the strategy could be implemented in practice. The test of the
  strategy should include the three elments in Exercise 1: plot, Sharpe
  ratio, and CAPM alpha. It is important to describe the machine
  learning model and the portfolio strategy sufficiently clearly that
  your results can be replicated.
\item
  Submit a Streamlit app and a Word doc. The app should have a Run
  button. When it is run, the app should download data, construct
  signals, and apply a pre-trained machine learning model to predict
  relative returns. The app should display the top 10 and bottom stocks
  and it should have a Download button that downloads an Excel workbook
  containing the following for every stock: ticker, all signals, and
  predicted relative return. The data should be sorted from best stock
  to worst stock. The Word doc should describe the machine learning
  model and it should describe a backtest of the model as in Exercise 2.
  It is important that the model and the backtest strategy be described
  sufficiently clearly that your results can be replicated.
\end{enumerate}

\paragraph{Course Schedule}\label{course-schedule}

\paragraph{CQA Competition}\label{cqa-competition}

Each year, the Chicago Quantitative Alliance (CQA) hosts a competition
for universities on quantitative investing - the
\href{https://www.cqa.org/investment_challenge}{CQA Challenge}. Twice
recently (2022 and 2023), a team of JGSB MBA students from this course
took first place, earning a prize of \$3,000.

The competition is to run a diversified long-short market-neutral
portfolio with a quantitative approach. Paper trading is done using the
StockTrak platform. Teams are judged on compliance, returns, and a video
presentation of their strategy in three stages: teams that perform well
on compliance proceed to the second stage, the top ten teams on
compliance and portfolio returns proceed to the third stage, and teams
in the third stage prepare video presentations.

The pipeline in Assignment 6 could be used to generate recommendations
for weekly trades. Unfortunately, the competition begins very shortly
after the class begins, so some stopgap trades will need to be put in
place before the pipeline is ready.

\paragraph{Honor Code}\label{honor-code}

The Rice University honor code applies to all work in this course. Use
of generative AI is of course permitted.

\paragraph{Disability Accommodations}\label{disability-accommodations}

Any student with a documented disability requiring accommodations in
this course is encouraged to contact me outside of class. All
discussions will remain confidential. Any adjustments or accommodations
regarding assignments or the final exam must be made in advance.
Students with disabilities should also contact Disability Support
Services in the Allen Center.




\end{document}
