\documentclass[aspectratio=169]{beamer}
\usetheme{metropolis}
\usepackage{xcolor}
\usepackage{listings}
\usepackage{hyperref}
\hypersetup{colorlinks=true, linkcolor=cyan, urlcolor=cyan, citecolor=cyan}

\title{MGMT 638}
\subtitle{Session 2: Stock Market Database \& Claude Code Setup}
\author{Kerry Back}
\institute{}
\date{Fall 2025}

\begin{document}

\maketitle


\begin{frame}{Database Guide - Complete Documentation}
\centering
\Large
\url{https://portal-guide.rice-business.org}

\vspace{2em}

\normalsize
Reference for all tables and variables in the database.
\end{frame}


\begin{frame}{Get an Access Token and view the Data Portal}

\begin{itemize}
\item Visit \href{https://data-portal.rice-business.org}{data-portal.rice-business.org}
\item Confirm your Rice email to receive an access token 
\end{itemize}
\end{frame}

\begin{frame}{How Does the Data Portal Work?}

\begin{itemize}
\item The data portal is an AI agent.  It is an app running in the cloud that has an API connection to ChatGPT (using my API key).
\item The database itself is stored at a different cloud location.
\item The app sends a system prompt explaining the database structure to ChatGPT along with the user prompt to get SQL code from ChatGPT.
\item The app then sends the SQL code to the database to get data.
\end{itemize}

\end{frame}
\begin{frame}{Claude as a General Agent}
\begin{itemize}
\item The recent trend is to configure Claude, ChatGPT, and Gemini as general agents, rather than creating separate apps for different purposes.
\item Model Context Protocol (MCP) introduced by Anthropic in Nov, 2024 was the first step.
\item It has been adopted by everyone.
\item The new method, introduced by Anthropic on Oct 16, 2025, is Claude Skills.
\end{itemize}
\end{frame}

\begin{frame}{Create .env File}

In VS Code:
\begin{itemize}
\item Tell Claude Code to create a \texttt{.env} file
\item Click the files icon in the left toolbar to open a file explorer tab
\item Open the \texttt{.env} file, and add this line: \texttt{RICE\_ACCESS\_TOKEN=your\_token\_here}
\item Replace \texttt{your\_token\_here} with your actual token.
\end{itemize}

\end{frame}

\begin{frame}{Download and Install Skill File}
\begin{itemize}
\item Ask Claude to create a folder: \texttt{.claude/skills/rice-data-query/}
\item Download \texttt{SKILL.md} from \url{https://mgmt638.kerryback.com/SKILL.md}
\item Save as \texttt{.claude/skills/rice-data-query/SKILL.md}
\item If SKILL.md opens in your web browser, right-click it and select 'Save As.'
\item Select File $\rightarrow$ Close Folder and then File $\rightarrow$ Open Folder and reopen your course folder.
\end{itemize}

\end{frame}

\begin{frame}[fragile]{Use Claude Code with Rice Data}

\textbf{Example prompts to try:}

\begin{itemize}
    \item Create a Jupyter notebook to get daily adjusted closing prices for AAPL since Jan 1, 2024 from the Rice database and plot them.
\item Create a Jupyter notebook to get ROE for TSLA by quarter on a trailing 4-quarters basis since the start of 2021 from the Rice database and plot it.
\item Get gross profits, total assets, ticker and date filed from all 10k's filed in 2025 from the Rice database and save the data as 10k\_2025.csv.
\end{itemize}

\end{frame}

\end{document}

\begin{frame}{What is Streamlit?}

\textbf{Streamlit} is a Python library for creating interactive web applications.

\vspace{1em}

\textbf{Why use Streamlit?}
\begin{itemize}
    \item Create data apps with pure Python (no HTML/CSS/JavaScript needed)
    \item Interactive widgets: sliders, dropdowns, text inputs, file uploads
    \item Real-time updates as users interact
    \item Display charts, tables, maps, and more
    \item Perfect for data exploration and visualization
\end{itemize}


\begin{frame}{Streamlit Example}

\textbf{Step 1: Create a Streamlit app file}

Ask Claude to create a Streamlit app that prompts the user to input their name and then opens up a new window saying Hello (name)! followed by a display of balloons.


\vspace{0.5em}

Ask Claude to run the app.
\end{frame}

\end{document}
