\documentclass[aspectratio=169]{beamer}
\usetheme{metropolis}
\hypersetup{colorlinks=true, linkcolor=cyan, urlcolor=cyan, citecolor=cyan}

\title{MGMT 638}
\subtitle{Session 5}
\author{Kerry Back}
\institute{}
\date{Fall 2025}

\begin{document}

\maketitle

\begin{frame}{Agenda}
\begin{enumerate}
\item Price and marketcap filters 
\item Standardizing predictors (features)
\item Composite ranks
\item Random forests
\item Standardizing returns
\item Safe minus risky strategy
\end{enumerate}
\end{frame}

\begin{frame}{Penny Stocks}
    \begin{itemize}
    \item Must filter out low-price stocks.  Infeasible for equally weighted portfolios and distort portfolio returns.
    \item Use CQA Competition filter: \$5.60?
    \item Example.  Use data from last class.  Tell Claude to read it and to drop all rows with close $\le$ price threshold.  Tell Claude to sort into quintiles each month on momentum and lagret and to compute the average return within each of the 25 groups each month.
    \end{itemize}
\end{frame}

\begin{frame}{Marketcap Filters}
    \begin{itemize}
    \item If we were managing a fund, we would specify in our prospective the universe of stocks in which we would invest.
    \item We would likely include some marketcap filter.  For example, large cap = S\&P 500, large and midcap = Russell 1000, smallcap = Russell 2000, etc.
    \item It is hard for large funds to trade microcaps, so they will exclude them even if the prospectus does not specify.
    \end{itemize}
\end{frame}

\begin{frame}{Standardizing Features}
\begin{itemize}
\item Many models, like linear regression, are highly affected by outliers.
\item Standardizing predictive variables is common.  For example, subtract mean and divide by standard deviation.  This is called a `z-score.'
\item Ranking 1 through $n$ is another way to standardize.  If $n$ varies, then can scale to 0 to 1 or (Quality Minus Junk), compute z-scores of ranks.
\item We should do this each month rather than for the full sample.
\item Important: For `good' features (like football scores), rank from low to high.  For `bad' features (like golf scores), rank from high to low.  Then a high rank is always a good value.
\end{itemize}
\end{frame}

\begin{frame}{Composite Ranks}
The Quality Minus Junk paper computes quality as follows.

\begin{enumerate}
\item Compute z-scores of features each month.
\item Group features into three categories: profitability, growth, and safety.
\item Within each category, add z-scores of features in that category each month.  Compute the z-score of the sum each month.
\item Add the three category z-scores together each month.  Compute the z-score of the sum each month.
\end{enumerate}
\end{frame}

\begin{frame}{Decision Trees}
\begin{itemize}
\item Start with the estimate $\hat y = \bar y$ for all observations.
\item Split into two subsets based on one variable and one threshold.  All observations below the threshold go into one group.  All above go into another.
\item Prediction for each group is the group mean of the target variable.  Calculate MSE (mean squared error) over both groups.
\item Choice of variable and threshold on which to split was based on minimizing MSE.
\item Split each subset into further subsets and continue.
\end{itemize}
\end{frame}

\begin{frame}{Example}
\begin{center}
\includegraphics[width=\textwidth]{Trees_Transparent.png}
\end{center}
\end{frame}

\begin{frame}{Small Example with Data}
    \begin{itemize}
    \item Tell Claude to use only the data for a single month.  Let's use Jan, 2018.
    \item Tell Claude to fit a decision tree with a maximum depth of three to predict the return from momentum, lagret, and book-to-market.
    \item Tell Claude to generate an image of the fitted tree.
    \end{itemize}
\end{frame}

\begin{frame}{Random Forests}
\begin{itemize}
\item A forest is multiple trees.
\item For any observation - old or new - each tree makes a prediction.
\item Average the predictions to get the final prediction.
\item A random forest is created by generating random datasets and fitting a tree to each.
\item A random dataset is generated by randomly drawing rows from the original dataset.
\end{itemize}
\end{frame}

\begin{frame}{Standardizing Returns}
    \begin{itemize}
    \item In most situations, it is uncommon to standardize the dependent variable.
    \item But if we want to pick stocks, we don't need to forecast returns.  We just need to forecast ranks.  Ranks are probably easier to forecast.
    \item So we could rank returns and compute z-scores of ranks each month to form our target variable.
    \end{itemize}
\end{frame}

\begin{frame}{Full Example: Fitting}
\begin{itemize}
\item Tell Claude to compute ranks of stocks each month on (i) momentum in ascending order, (ii) lagret in descending order, and (iii) book-to-market in ascending order.
\item Tell Claude to compute ranks of stocks each month on return in ascending order.
\item Tell Claude to compute z-scores of the ranks each month.
\item Tell Claude to fit a random forest using data through 2015 to predict the z-score of return ranks from the z-scores of the feature ranks.  Tell Claude to use 100 trees and a maximum depth of 4.
\end{itemize}
\end{frame}

\begin{frame}{Full Example: Predicting and Evaluating}
\begin{itemize}
\item Tell Claude to use the fitted model to predict the target variable for each stock each month from 2016 on.
\item Tell Claude to sort stocks into deciles each month based on the predicted values.
\item Tell Claude to compute the average return of each group each month.
\item Tell Claude to compute the mean , standard deviation, and mean-to-standard-deviation ratio of each decile return series.
\end{itemize}
\end{frame}

\begin{frame}{Safe Minus Risky}
\begin{itemize}
\item \href{https://mgmt638.kerryback.com/pdfs/Kapadia_Ostdiek_Weston_Zekhnini_JFQA_2019.pdf}{PDF: \textit{Safe Minus Risky}, Kapadia, Ostdiek, Weston, and Zekhnini, 2019}
\item \href{https://mgmt638.kerryback.com/videos/Kapadia_Ostdiek_Weston_Zekhnini_JFQA_2019.mp4}{Video: \textit{Safe Minus Risky}, Kapadia, Ostdiek, Weston, and Zekhnini, 2019}
\item \href{https://notebooklm.google.com/notebook/58351fb6-7453-4bff-909f-281bb34e4c6a}{NotebookLM: Safe Minus Risky}
\end{itemize}
\end{frame}

\begin{frame}{Industry Betas and Returns}
\centering
\Large
\url{https://learn-investments.rice-business.org/capm/sml-industries}
\end{frame}

\end{document}
