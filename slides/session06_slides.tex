\documentclass[aspectratio=169]{beamer}
\usetheme{metropolis}
\hypersetup{colorlinks=true, linkcolor=cyan, urlcolor=cyan, citecolor=cyan}

\title{MGMT 638}
\subtitle{Session 6}
\author{Kerry Back}
\institute{}
\date{Fall 2025}

\begin{document}

\maketitle

\begin{frame}{Agenda}
    \begin{enumerate}
    \item Hedge funds and public information paper 
    \item More random forests
    \item Creating Quality Minus Junk characteristics 
    \end{enumerate}
\end{frame}

\begin{frame}{Hedge Funds and Public Information}
\begin{itemize}
\item \href{https://mgmt638.kerryback.com/pdfs/Crane_Crotty_Umar_MS_2023.pdf}{PDF: \textit{Hedge Funds and Public Information}, Crane, Crotty, and Umar, 2023}
\item \href{https://mgmt638.kerryback.com/videos/Crane_Crotty_Umar_MS_2023.mp4}{Video: \textit{Hedge Funds and Public Information}, Crane, Crotty, and Umar 2023}
\item \href{https://notebooklm.google.com/notebook/93761776-77e8-45ae-a5fa-3010fdf23d65}{NotebookLM}
\end{itemize}
\end{frame}

\begin{frame}{More Random Forests}
\begin{itemize}
\item Ask Claude to get the California housing price dataset and to describe it.
\item Ask Claude to do a train/test split and fit a random forest on the training data to predict house prices.
\item Ask Claude to evaluate performance on the test data.
\item Ask Claude to vary the maximum depth and to display performance on the training and test data.
\end{itemize}
\end{frame}
\begin{frame}{Create Quality Minus Junk Variables}

    \begin{itemize}
    \item See Session1/Quality Minus Junk for link to paper
    \item See Session3/Workflow for procedure to follow
    \item Important: create all growth rates of ratios or other fundamental variables before merging with returns
    \end{itemize}

\end{frame}

\end{document}
\section{Claude Agent SDK}

\begin{frame}{Creating an Agentic Application}
\textbf{Goal:} Build a web app where users request financial data histograms and watch the AI agent iteratively solve problems in real-time.

\vspace{0.3cm}
\textbf{Key Features:}
\begin{itemize}
\item Agent runs in a browser tab
\item Input box for characteristic
\item Agent queries Rice Database with SQL and gets most recent values
\item Generates histogram from characteristic values
\item Displays the histogram in the app
\end{itemize}
\end{frame}

\begin{frame}{Windows Subprocess Challenge}
\textbf{Problem:} Claude Agent SDK spawns \texttt{claude} CLI as subprocess, but Windows ProactorEventLoop doesn't support subprocess creation.

\vspace{0.3cm}
\textbf{Error:}
\begin{itemize}
\item \texttt{NotImplementedError} in \texttt{asyncio.base\_events.py:535}
\item Even with \texttt{WindowsSelectorEventLoopPolicy()}
\item Problem: SDK looks for \texttt{claude} via \texttt{shutil.which()}, but finds Unix shell script instead of \texttt{claude.exe} on Windows.

\vspace{0.3cm}
\textbf{Solution:} Use Anthropic API directly with agentic tool-calling loop (same functionality, no subprocess issues).
\end{frame}

\begin{frame}[fragile]{Working Solution: Agentic Tool-Calling Loop}
\small
Core pattern using Anthropic API:
\begin{verbatim}
client = anthropic.Anthropic(api_key=os.getenv('ANTHROPIC_API_KEY'))
messages = [{"role": "user", "content": prompt}]

for turn in range(max_turns):
    response = client.messages.create(
        model="claude-sonnet-4-5-20250929",
        system=SYSTEM_PROMPT,
        messages=messages,
        tools=[{"name": "bash", "description": "...", "input_schema": {...}}]
    )

    # Stream agent's thinking to browser
    for block in response.content:
        if block.type == "text":
            yield f"💭 {block.text[:300]}"
\end{verbatim}
\end{frame}

\begin{frame}[fragile]{Tool Execution and Error Handling}
\footnotesize
\begin{verbatim}
    if response.stop_reason == "tool_use":
        tool_results = []
        for block in response.content:
            if block.type == "tool_use":
                cmd = block.input["command"]
                yield f"🔧 Running: {cmd[:200]}"

                result = subprocess.run(cmd, shell=True,
                                       capture_output=True, timeout=120)
                output = result.stdout if result.returncode == 0
                         else f"Error: {result.stderr}"

                yield f"✅ Result: {output[:500]}"

                tool_results.append({
                    "type": "tool_result",
                    "tool_use_id": block.id,
                    "content": output
                })

        messages.append({"role": "user", "content": tool_results})
\end{verbatim}
\end{frame}

\begin{frame}{Rice Database Knowledge Integration}
\textbf{System Prompt} includes rice-data-query skill information:

\vspace{0.3cm}
\begin{itemize}
\item API endpoint: \texttt{https://data-portal.rice-business.org/api/query}
\item Authentication using \texttt{RICE\_ACCESS\_TOKEN}
\item Table schemas: DAILY (pe, pb, marketcap), SEP (prices), SF1 (fundamentals)
\item Critical rules:
\begin{itemize}
    \item All dates are VARCHAR - must cast to DATE
    \item Market cap values in thousands
    \item Most recent data: \texttt{WHERE date = (SELECT MAX(date) FROM table)}
    \item Filter NULL and extreme outliers
\end{itemize}
\end{itemize}

\vspace{0.2cm}
\textbf{Result:} Agent knows how to query database correctly without trial-and-error.
\end{frame}

\begin{frame}{Real-Time Streaming with Server-Sent Events}
\textbf{FastAPI endpoint} streams progress to browser:

\vspace{0.3cm}
{\small
\texttt{@app.get("/generate\_stream")}\\
\texttt{async def generate\_stream(characteristic: str):}\\
\quad\texttt{return StreamingResponse(}\\
\quad\quad\texttt{run\_agent\_with\_streaming(characteristic),}\\
\quad\quad\texttt{media\_type="text/event-stream"}\\
\quad\texttt{)}
}

\vspace{0.3cm}
\textbf{JavaScript} in browser receives updates:
\begin{itemize}
\item \texttt{\{"type": "thinking", "content": "I'll query DAILY table..."\}}
\item \texttt{\{"type": "tool", "content": "python query\_pe.py"\}}
\item \texttt{\{"type": "tool\_result", "content": "Error: column not found"\}}
\item \texttt{\{"type": "thinking", "content": "Let me revise the query..."\}}
\end{itemize}
\end{frame}

\begin{frame}{Proof of Concept Demo}
\textbf{File:} \texttt{rice\_histogram\_final.py}

\vspace{0.3cm}
\textbf{Run:} \texttt{python rice\_histogram\_final.py}

\vspace{0.3cm}
\textbf{Open browser:} \texttt{http://localhost:8005}

\vspace{0.3cm}
\textbf{Try:} "market cap" or "PE ratio"

\vspace{0.5cm}
\textbf{What you'll see:}
\begin{enumerate}
\item Agent writes Python code to query database
\item Code is sent to Rice Data Portal
\item Error message returned (e.g., wrong table/column)
\item Agent debugs and revises code
\item Histogram created with summary statistics
\end{enumerate}

\vspace{0.3cm}
\textbf{Key Achievement:} Real-time visibility into agentic problem-solving process.
\end{frame}

\begin{frame}{Resources: Claude Agent SDK}
\textbf{Official Anthropic Documentation:}

\vspace{0.3cm}
\begin{itemize}
\item \href{https://docs.claude.com/en/docs/agent-sdk/overview}{Agent SDK Overview}
    \begin{itemize}
    \item Complete documentation and getting started guide
    \item Installation instructions for Python and TypeScript
    \end{itemize}

\vspace{0.2cm}
\item \href{https://www.anthropic.com/engineering/building-agents-with-the-claude-agent-sdk}{Building Agents with the Claude Agent SDK}
    \begin{itemize}
    \item Engineering blog post with detailed examples
    \item Best practices for building production agents
    \end{itemize}

\vspace{0.2cm}
\item \href{https://github.com/anthropics/claude-agent-sdk-python}{GitHub: claude-agent-sdk-python}
    \begin{itemize}
    \item Official Python SDK repository
    \item Code examples and API reference
    \end{itemize}
\end{itemize}

\vspace{0.3cm}
\textbf{Note:} Our implementation uses the Anthropic API directly with tool calling, which provides the same agentic behavior without Windows subprocess limitations.
\end{frame}

\end{document}
