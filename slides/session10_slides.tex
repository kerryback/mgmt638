\documentclass[aspectratio=169]{beamer}
\usetheme{metropolis}
\hypersetup{colorlinks=true, linkcolor=cyan, urlcolor=cyan, citecolor=cyan}

\title{MGMT 638}
\subtitle{Session 10}
\author{Kerry Back}
\institute{}
\date{Fall 2025}

\begin{document}

\maketitle



\begin{frame}{Agenda}
\begin{itemize}
\item Evaluating portfolio performance
\begin{itemize}
\item Mean, standard deviation, and Sharpe ratio
\item Drawdowns
\item Alphas
\item Attribution analysis
\item Comprehensive evaluation notebook
\end{itemize}
\item Weekly trading
\begin{itemize}
\item Data (and new skills)
\item Train-test script 
\item Results
\end{itemize}
\item Trading costs 
\end{itemize}
\end{frame}

\begin{frame}[c]
\centering
\Huge Evaluating Portfolio Performance
\end{frame}

\begin{frame}{Alphas and Portfolio Improvement}
Suppose you're in wealth management.  Should you (and your client) care about alphas?  If so, what alpha?

\vspace{0.5cm}

\href{https://learn-investments.rice-business.org/capm/alphas-mve}{Learn Investments page}
\end{frame}

\begin{frame}{Ask Claude}
    \begin{enumerate}
    \item The risk-free rate is 2\%.  My portfolio has an expected return of 10\% and a standard deviation of 15\%.  I'm considering moving some money to a new asset that has a standard deviation of 30\%, a correlation with my portfolio of 60\%, and an alpha relative to my portfolio of 5\%.  How can I combine the new asset and the risk-free asset with my current portfolio to get an expected return higher than 10\% while maintaining the standard deviation at 15\%?
    \item Put this analysis in a Jupyter notebook, doing the calculations in Python.
    \end{enumerate}

\end{frame}

\begin{frame}{When Does a New Asset Have Positive Alpha?}
A new asset has positive alpha relative to the current portfolio if and only if:

%\vspace{0.1cm}

$$\alpha > 0 \quad \Longleftrightarrow \quad \text{Sharpe}_{\text{new}} > \rho \times \text{Sharpe}_{\text{current}}$$

%\vspace{0.1cm}

where:
\begin{itemize}
\item $\text{Sharpe}_{\text{new}} = \frac{E(R_{\text{new}}) - r_f}{\sigma_{\text{new}}}$ is the Sharpe ratio of the new asset
\item $\text{Sharpe}_{\text{current}} = \frac{E(R_{\text{current}}) - r_f}{\sigma_{\text{current}}}$ is the Sharpe ratio of the current portfolio
\item $\rho$ is the correlation between the new asset and current portfolio
\end{itemize}

\vspace{0.3cm}

\textbf{Key insight:} Lower correlation makes it easier for an asset to have positive alpha.
\end{frame}

\begin{frame}{Attribution Analysis}
\begin{itemize}
\item Suppose a strategy has a positive alpha.  We will want to know how the manager is getting it.  \textbf{What risks is the manager taking on to produce alpha?}
\item Can start with the Fama-French factors: SMB, HML, CMA, RMW.
\item Is the manager betting on size?  On value?  On conservative stocks?  On profitable stocks?
\item Run regression of excess returns on 5 factors (including Mkt-RF).  Analyze beta coefficients.
\item Can add more factors: momentum, volatility, liquidity, \ldots
\end{itemize}
\end{frame}

\begin{frame}{New Evaluation Notebook}

\href{https://mgmt638.kerryback.com/analyze_portfolios.ipynb}{analyze\_portfolios.ipynb}

\vspace{.5cm}

Note: You can ask Claude to create a PowerPoint deck from the tables and figures produced in the notebook.
\end{frame}


\begin{frame}[c]
\centering
\Huge Weekly Data
\end{frame}


\begin{frame}{Updated Skills}
\begin{enumerate}
\item Replace your .claude folder with the .claude folder in  \href{https://mgmt638.kerryback.com/claude.zip}{this zip file}.
\item data5.parquet was created with this prompt: 
\begin{quote}
Use the Rice database to create a weekly dataset containing the variables in data4.parquet, but compute lag\_week and lag\_month instead of lagged\_return.  I want all stocks and all dates after Jan 1, 2010.  Read the weekly dataset skill in its entirety and follow its instructions exactly.  If anything is unclear, please ask for clarification.
\end{quote}
\end{enumerate}
\end{frame}


\begin{frame}{Data, Train/Test Script, and Outputs}
Download \href{https://mgmt638.kerryback.com/PLACEHOLDER.zip}{\texttt{data5\_pipeline.zip}}.  It contains:
\begin{itemize}
\item data5.parquet (weekly version of data4.parquet)
\item train\_predict\_data5.py (backtesting script for weekly data)
\item outputs of the backtesting script
\begin{itemize}
\item data5\_predict.parquet - (ticker, week, predict) predictions
\item data5\_portfolios.csv - (week, decile, return, predict) portfolio analysis
\item data5\_current.xlsx - Current week predictions with features
\item data5\_model.pkl - Trained LightGBM model
\end{itemize}
\end{itemize}
\end{frame}


\begin{frame}{Weekly Train-Predict Script}
\begin{itemize}
\item All features and return converted to percentile ranks (0--1)
\item Train on most recent 52 weeks
\item Use each trained model for 8 weeks, then retrain
\item Out-of-sample predictions only (no look-ahead bias)
\item 52 and 8 are adjustable parameters.  Could train every week, but would be slower.
\end{itemize}
\end{frame}

\begin{frame}{Results}
\begin{itemize}
\item 2.35M predictions across 725 weeks (2012-02 to 2025-48)
\item Average weekly spread (D10 - D1): 0.60\%
\item \texttt{data5\_predict.parquet} - All predictions
\item \texttt{data5\_portfolios.csv} - Decile analysis by week
\end{itemize}
\end{frame}

\begin{frame}[c]
\centering
\Huge Trading Costs
\end{frame}

\begin{frame}{Categories of Costs}
    \begin{itemize}
    \item Commissions
    \item Bid-ask spread
    \item Price impacts
    \item Opportunity cost of trades cancelled due to price impacts
    \end{itemize}
\end{frame}

\begin{frame}{Cost Management}
\begin{itemize}
\item Trade gradually in small amounts 
\item Use limit orders instead of or in addition to market orders
\item Monitor the market and the limit-order book to decide when to submit and cancel orders (algorithmic trading)
\end{itemize}
\end{frame}

\begin{frame}{Frazzini-Israel-Moskowitz, 2012}
\begin{itemize}
\item \href{https://mgmt638.kerryback.com/pdfs/Frazzini_Israel_Moskowitz.pdf}{PDF: \textit{Trading Costs of Asset Pricing Anomalies}: Frazzini, Israel, and Moskowitz, 2012}
\item \href{https://mgmt638.kerryback.com/pdfs/Frazzini_Israel_Moskowitz.mp4}{Video: \textit{Trading Costs of Asset Pricing Anomalies}: Frazzini, Israel, and Moskowitz, 2012}
\item \href{https://notebooklm.google.com/notebook/b937c926-b872-47b0-a85d-43faf5d7944d}{NotebookLM}
\end{itemize}
\end{frame}
\end{document}
