\documentclass[aspectratio=169]{beamer}
\usetheme{metropolis}
\hypersetup{colorlinks=true, linkcolor=cyan, urlcolor=cyan, citecolor=cyan}

\title{MGMT 638}
\subtitle{Session 11}
\author{Kerry Back}
\institute{}
\date{Fall 2025}

\begin{document}

\maketitle

\begin{frame}{Agenda}
\begin{itemize}
\item Model interpretability 
\item Neural networks
\end{itemize}
\end{frame}

\begin{frame}{Model Intrepretability}
\begin{itemize}
\item Use lightGBM trained on all monthly data 2011-02 through 2025-10: \href{https:mgmt638.kerryback.com/lightGBM.pkl}{lightGBM.pkl}
\item Use 2025-11 features: \href{https://mgmt638.kerryback.com/currentdata.xlsx}{currentdata.xlsx}
\item Why is the model making the predictions that it is making in 2025-11?
\end{itemize}

\vspace{0.5cm}
\href{https://mgmt638.kerryback.com/session11A_interpretability.ipynb}{Interpretability notebook}
\end{frame}

\begin{frame}{Neural Networks}
\begin{itemize}
\item Model structure
\item Classification vs regression
\item Digits example
\item Confusion matrix
\item Increasing complexity and performance
\end{itemize}

\vspace{0.5cm}
\href{https://mgmt638.kerryback.com/session11B_neuralnetworks.ipynb}{Neural networks notebook}
\end{frame}

\begin{frame}{Another Example}
Tell Claude:

\vspace{0.3cm}
\begin{quote}
Get the breast cancer dataset.  Explain it, including how the target variable is coded.  Train a neural network and evaluate it on train and test sets.  Display the confusion matrix for the test set.
\end{quote}
\end{frame}
\end{document}  